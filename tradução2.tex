\documentclass[a4paper, 12 pt]{article}
\usepackage[top=2cm, bottom=2cm, left=2.5cm, right=2.5cm]{geometry}
\usepackage[utf8]{inputenc}
\begin{document}
A última década testemunhou uma adoção sem precedentes do aprendizado de máquina
técnicas para dar sentido aos dados disponíveis e fazer previsões para apoiar a
tomada de decisão para uma ampla variedade de aplicações, desde análises de saúde
ics para previsões de rotatividade de clientes, recomendações de filmes e política macroeconômica. O foco na literatura de aprendizado de máquina tem sido cada vez mais
sistemas sofisticados com o objetivo primordial de melhorar a precisão de seus
previsões ao custo de tornar esses sistemas essencialmente caixa-preta. Enquanto em
certas tarefas, como previsões de anúncios, a precisão é o objetivo principal, em outros
domínios, por exemplo, legal, médico e governamental, é essencial que o
tomadores de decisão que podem não ter sido treinados em aprendizado de máquina podem interpretar
e validar as previsões.
As tecnicas interpretaveis ​​mais populares que tendem a ser adotadas e
confiaveis ​​pelos tomadores de decisão incluem regras de classificação, árvores de decisão e decisões
listas de ação [10,29,8,30]. Em particular, regras de decisão com um pequeno número de booleanos
as clausulas tendem a ser mais interpretaveis. Esses modelos podem ser usados ​​para aprender
modelos interpretáveis ​​desde o início, e também como proxies que fornecem post-hoc
explicações para modelos de caixa preta pré-treinados [12,1].
Na frente teórica, o problema de aprendizagem de regras mostrou ser com-
putacionalmente intratável [32]. Consequentemente, os primeiros esforços práticos, como
as abordagens de lista de decisão e árvore de decisão se basearam em uma combinação de heuristicamente
objetivos de otimização escolhidos e técnicas algorítmicas gananciosas, e o tamanho
da regra foi controlada por parada antecipada ou poda de regra ad-hoc. Somente
recentemente tem havido algumas formulações que tentam equilibrar o acúmulo
atrevido e o tamanho da regra em um objetivo de otimização de princípio, seja por meio
otimização combinatória, relaxamentos de programação linear (LP), submodular
otimização, ou métodos bayesianos [4,25,24] [7,34], conforme revisamos na Seção 5.
Motivado pelo progresso significativo no desenvolvimento de sistemas combinatórios
solvers (em particular, MaxSAT), perguntamos: podemos projetar um quadro combinatório
trabalhar para construir com eficiência regras de classificação interpretáveis ​​que aproveitem
desses avanços recentes? A principal contribuição deste artigo é apresentar
uma estrutura combinatória que permite um controle preciso de precisão vs.
pretabilidade, e verificar se os avanços computacionais na comunidade MaxSAT
pode tornar prático a solução de problemas de classificação em grande escala.
Em particular, este documento faz as seguintes contribuições:
1. Uma estrutura baseada em MaxSAT, MLIC, que comprovadamente negocia precisão vs.
interpretabilidade das regras
2. Uma implementação de protótipo de MLIC baseada em MaxSAT que é capaz
de encontrar regras de classificação ótimas (ou de alta qualidade quase ótimas) de
conjuntos de dados modernos em grande escala
3. Mostramos que em muitos problemas de classificação a interpretabilidade pode ser alcançada
com apenas uma pequena perda de precisão e, além disso, MLIC, que especificamente
procura regras interpretáveis, pode aprender com muito menos amostras do que
técnicas de caixa de ML.
Além disso, esperamos compartilhar nosso entusiasmo com aplicações de restrição
programação / MaxSAT em Aprendizado de Máquina, e para encorajar pesquisadores em
classificação interpretável e nas comunidades CSP / SAT a considerar
este tópico mais adiante: tanto no desenvolvimento de novas formulações baseadas em SAT para inter
pretable ML, e na concepção de solucionadores sob medida sintonizados com o problema de inter
pretable ML.
O resto do artigo está organizado da seguinte forma: Discutimos notações e prelim-
inários na Seção 2. Em seguida, apresentamos o MLIC, que é a principal contribuição
deste artigo, na Seção 3 e acompanhamento com configuração experimental e resultados
em um grande conjunto de benchmarks na Seção 4. Discutimos então o trabalho relacionado em
Seção 5 e finalmente concluir na Seção
\end{document}
\documentclass[a4paper, 12 pt]{article}
\usepackage[top=2cm, bottom=2cm, left=2.5cm, right=2.5cm]{geometry}
\usepackage[utf8]{inputenc}
\begin{document}
O aprendimento de máquinas testemunhou um grande progresso e importante sucesso nos anos recentes. Em algumas configurações, previsões feitas por algoritmo de aprendizagem de máquinas devem promover explicações que preferivelmente pode interpretar ou serem entendidas por marcadores de decisão humanas. Exemplos concretos incluem situações de segurança critica, mas também quando decisões de transparência são superiores. A importância da inteligencia artificial que pode ser explicado, exemplo: o problema de explicações associáveis com as previsões ML é enfatizado em pesquisas recentes por programas de pesquisas em andamento, pelo nível de legislação EU pelo qual é esperado por reforçar a geração automática de explicações, e também o número de encontros de modelos de computação explicáveis ML.
	
	Um uso frequente utilizado para promover explicações para as previsões ML é recorrer a um tipo de modelo logica relacional, incluindo listas de regras/decisões, Conjuntos de regras/decisões, e arvores de decisão. Esse modelo de logica relacional pode na maioria dos casos associar explicações com previsões,representadas como conjunções de literais, que vem de um modelo real de representação. Claramente, um menor modelo de representação, as explicações tendem a ser mais simples, e também mais fáceis de serem entendidas por marcadores de decisões humanas.
	
	Abordagens recentes incluem a computação de listas de regras(maiores ou menores), a computação de conjuntos de decisão, e também a computação de arvore de de decisão. Lista de regras impõem uma ordem de regras, enquanto conjunto de decisão não. Claramente por uma perspectiva interpretável, conjuntos de decisão seçao mais atrativos uma ve que cada previsão depende apenas as associações literais de cada regra. Pelo lado negativo conjuntos de regra exibem uma sobre posição de regras, por tanto podem requerer que decisões sejam feitas quando mais de uma classe é prevista. Além disso, mesmo formas restritas de aprendizagem de regras são conhecidas por serem difíceis por NP.
	
	Esse artigo analisa o trabalho recente em conjuntos de computação interpretáveis. Esse artigo destaca o número de desvantagens da abordagem proposta, relaciona cm sobreposição de regras, a geração de explicações, e também a possibilidade de alterara a escala da aplicação proposta. Esse texto portanto investiga tres tópicos principais. O primeiro tópico é a proposta de definições rigorosas de sobreposição de regras. Esss texto se relaciona essa nova definição com o trabalho recente, e com conjunturas de que resolvendo o problema da sobreposição quando aprendendo o conjunto de decisões mais adequado é dificil para o segundo nível da hierarquia polinomial. Este artigo então propoe um numero de variantes de conjunto de decisões de aprendizagem com uma demanda de limitações menor na sobreposições, e mostra que essas variantes são realmente difíceis para NP. O segundo tópico é problema de explicações geradas para previsões. Esse artigo mostra diferentes modelos para aprendizagem de conjuntos de decisão providas por diferentes formas de explicações de computação, dessa forma dando capacidade a explicações de geração na maioria das configurações. O terceiro tópico é desenvolver modelos proposicionais diferente para aprendizagem com um dos mais eficientes conjuntos de decisão. Os modelos propostos criando em um trabalho recente com inferência indutiva, mas introduzem um numero de variantes, permitindo múltiplas classes, e também acomodando diferente limites de sobreposição. Além disso, esse artigo mostra que todos esses modelos exibem simetrias na formulações de problemas, e então predicações quebrando essas simetrias podem ser usadas para melhorar a performance.
	
	Esse artigo é organizado então dessa forma. Seção 2 introduz as definições e anotações usadas no restante do texto. O problema da sobreposição e explicação da gerações é investigada na seção 3. Modelos de oferta para aprendizagem de conjuntos de decisão sujeitos a diferentes limites na sobreposição são propostos na seção 4. A seção 5 analisa a performance da abordagem proposta nos respectivos conjuntos de dados, e compara com trabalhos recentes. A seção 6 conclui o artigo.
\end{document}
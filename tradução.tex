\documentclass[a4paper, 12 pt]{article}
\usepackage[top=2cm, bottom=2cm, left=2.5cm, right=2.5cm]{geometry}
\usepackage[utf8]{inputenc}
\begin{document}
O aprendizado de máquina (ML) testemunhou um progresso notável e sucessos importantes
nos últimos anos. Em algumas configurações, as previsões feitas por algoritmos de aprendizado de máquina
deve fornecer explicações, de preferência explicações que possam ser interpretadas
(ou compreendido) por tomadores de decisão humanos. Exemplos concretos incluem segurança crítica
situações, mas também quando a transparência das decisões é fundamental. A importância de
AI explicável (XAI), ou seja, o problema de associar explicações a previsões de ML,
é sublinhado por pesquisas recentes [2, 21, 42], por programas de pesquisa em andamento [9],
pela legislação a nível da UE que se espera que imponha a geração automatizada de explicações
[11], e também por uma série de reuniões sobre a computação de modelos de ML explicáveis
[16, 17, 30].
Uma abordagem frequentemente usada para fornecer explicações para as previsões de ML é recorrer a
algum tipo de modelo relacionado à lógica, incluindo listas de regras / decisões, conjuntos de regras / decisões e
árvores de decisão [2, 21]. Esses modelos relacionados à lógica podem, na maioria dos casos, associar explicações
com previsões, representadas como conjunções de literais, que decorrem da representação real do modelo. Claramente, quanto menor for a representação do modelo, mais simples
as explicações provavelmente serão e, portanto, mais fáceis de entender pelos tomadores de decisão humanos.
Abordagens recentes incluem o cálculo de listas de regras (menores ou menores) [2,42],
o cálculo de conjuntos de decisão [21], mas também o cálculo de árvores de decisão [3].
As listas de regras impõem uma ordem das regras [35], enquanto os conjuntos de decisões não. Claramente,
de uma perspectiva de interpretabilidade, os conjuntos de decisões são os mais atraentes, uma vez que cada
a previsão depende apenas dos literais associados a cada regra. Do lado negativo,
conjuntos de decisões podem exibir sobreposição de regras e, portanto, podem exigir que decisões sejam tomadas quando
mais de uma classe está prevista. Além disso, mesmo as formas restritas de aprendizado de regras são
bem conhecido por ser difícil para NP [35] 
Este artigo analisa trabalhos recentes sobre computação de conjuntos de decisões interpretáveis. O
artigo destaca uma série de desvantagens da abordagem proposta, relacionadas com a regra
se sobrepõem, a geração de explicações, mas também com a escalabilidade da abordagem.
O artigo então investiga três tópicos principais. O primeiro tópico é a proposta de um rigoroso
definição de sobreposição de regra. O artigo relaciona esta nova definição com trabalhos anteriores, e
conjectura que resolver o problema de sobreposição ao aprender a decisão ideal (em tamanho)
conjuntos é difícil para o segundo nível da hierarquia polinomial. O artigo então propõe um
número de variantes de conjuntos de decisões de aprendizagem com restrições menos exigentes na sobreposição,
e mostra que essas variantes são difíceis para NP. O segundo tópico é o problema
de gerar explicações para previsões. O artigo mostra que diferentes modelos para
conjuntos de decisão de aprendizagem fornecem diferentes formas de explicações computacionais, permitindo assim
a geração de explicações na maioria dos ambientes. O terceiro tópico é desenvolver diferentes
modelos proposicionais para aprender conjuntos de decisão ótimos. Os modelos propostos baseiam-se
trabalhos anteriores sobre inferência indutiva [19], mas introduzem uma série de variantes, permitindo
para várias classes e também acomodando diferentes restrições de sobreposição. 
O artigo está organizado da seguinte forma. Seção 2 apresenta as definições e notação
usado no restante do artigo. A questão da sobreposição e geração de explicação
é investigado na Seção 3. Modelos proposicionais para conjuntos de decisões de aprendizagem sujeitos a
diferentes restrições sobre a sobreposição são propostas na Seção 4. A Seção 5 analisa o desempenho
da abordagem proposta em conjuntos de dados representativos, e compara com anteriores
trabalho [21]. A seção 6 conclui o artigo
\end{document}
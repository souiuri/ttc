
\begin{document}
7O mundo mudou com os avanços rápidos e recentes no aprendizado de máquina. Decisão
tarefas que pareciam muito além das capacidades da inteligência artificial agora se tornaram
comumente resolvido usando aprendizado de máquina [32
, 35
, 41]. Mas isso tem algum custo.
A maioria dos algoritmos de aprendizado de máquina são opacos, incapazes de explicar por que as decisões foram
feito. Pior, eles podem ser influenciados por seus dados de treinamento e se comportar mal quando expostos
a dados fora dos quais foram treinados. Daí o crescente interesse em explicável
inteligência artificial (XAI) [
4
14
, 16
, 18
, 22
, 23
, 26
, 29
30
, 36
–38
, 42
, 43
, 48
, 49
, 52],
incluindo programas de pesquisa [
3
, 24] e legislação [17
, 21].
Neste artigo, vamos nos concentrar em problemas de classificação, onde a entrada é um conjunto de
instâncias com recursos e, como um rótulo, uma classe para prever. Para esses problemas, alguns dos
as formas mais explicáveis ​​de formalismo de aprendizado de máquina são conjuntos de decisões
[11
, 12
,
15
, 20
31
, 34
, 39]. Um conjunto de decisão é um conjunto de regras de decisão, cada uma com condições
C e
decisão
X, de modo que se uma instância satisfizer
C, então sua classe está prevista para ser
X. An
vantagem dos conjuntos de decisão sobre as árvores de decisão e listas de decisão mais populares é que
cada regra pode ser entendida de forma independente, tornando este formalismo um dos mais fáceis
explicar. Na verdade, a fim de explicar uma decisão particular na instância
D, podemos apenas
referem-se a uma única regra de decisão
C
⇒
X s.t.
D satisfaz
C
.
Para que os conjuntos de decisões sejam claros e explicáveis ​​a um ser humano, as regras individuais devem ser
conciso. Trabalhos anteriores examinaram conjuntos de decisões de construção que envolvem o mínimo
regras possíveis e, em seguida, minimiza o número de literais nas regras [31]; ou construção
arXiv: 2007.15140v1 [cs.AI] 29 de julho de 2020
2 Yu et al.
um classificador CNF que fixa o número de regras e, em seguida, minimiza o número de literais [20, 39] para explicar as instâncias positivas da classe. Este trabalho também sofre de
a limitação de que as regras apenas prevêem a classe 1, e o modelo prevê a classe 0 se não
regra se aplica. Infelizmente, para explicar uma instância de classe 0, precisamos usar (o
negação de todas as regras, tornando as explicações não sucintas.
Neste trabalho, argumentamos que o número de regras é a medida errada de explicabilidade,
visto que, por exemplo, 3 regras, cada uma envolvendo 100 condições, são provavelmente menos compreensíveis do que, digamos, 5 regras, cada uma envolvendo 20 condições. Na verdade, desde a explicação
de uma única instância é apenas uma única regra de decisão, o número de regras está longe de ser
tão importante quanto o tamanho das regras individuais. Portanto, o trabalho anterior sobre a construção de conjuntos de decisão de tamanho mínimo não usou a melhor medida de tamanho para explicabilidade.
Neste trabalho, examinamos a construção direta de conjuntos de decisão do menor total
tamanho, onde o tamanho de uma regra com as condições C é | C | + 1 (o 1 adicional é para o
descritor de classe X). Isso leva a conjuntos de decisões menores (em termos de literais) que parecem
muito mais atraente para explicar decisões.
Acontece que esta definição de tamanho leva a modelos SAT que são experimentalmente
mais difícil de resolver, mas os conjuntos de decisões resultantes podem ser significativamente menores. Contudo,
para conjuntos de decisões esparsos, onde podemos considerar um conjunto de regras menor, se houver
não cometer muitos erros na classificação, esta nova medida não é mais difícil de
compute do que a medida de contagem de regra tradicional e fornece decisões de granularidade mais precisas
na dispersão.
As contribuições deste artigo são
- A primeira abordagem para construir conjuntos de decisão ideais em termos do número total de
literais necessários para definir todo o conjunto,
- Modelos SAT e MaxSAT alternativos para resolver este problema e variações esparsas
que permitem uma troca de precisão versus tamanho,
- Resultados experimentais detalhados que mostram a aplicabilidade desta abordagem, que
demonstrar que nossa melhor abordagem pode gerar conjuntos de decisão esparsos ideais rapidamente
com precisão comparável aos melhores métodos heurísticos, mas muito menor.
O artigo está organizado da seguinte forma. Seção 2 apresenta a notação e definições
usado em todo o papel. O trabalho relacionado é descrito na Seção 3. A Seção 4 descreve
as novas codificações baseadas em SAT e MaxSAT para a inferência de conjuntos de decisão. Os resultados experimentais são analisados ​​na Seção 5. Finalmente, a Seção 6 conclui o artigo.
\end{document}